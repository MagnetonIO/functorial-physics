\documentclass[11pt,letterpaper]{article}
\usepackage[utf8]{inputenc}
\usepackage{amsmath,amssymb,amsthm}
\usepackage{graphicx}
\usepackage{hyperref}
\usepackage{physics}
\usepackage{cite}
\usepackage{geometry}
\geometry{margin=1in}

% Theorem environments
\newtheorem{theorem}{Theorem}[section]
\newtheorem{lemma}[theorem]{Lemma}
\newtheorem{proposition}[theorem]{Proposition}
\newtheorem{corollary}[theorem]{Corollary}
\newtheorem{definition}[theorem]{Definition}
\newtheorem{remark}[theorem]{Remark}

% Commands
\newcommand{\R}{\mathbb{R}}
\newcommand{\Z}{\mathbb{Z}}
\newcommand{\N}{\mathbb{N}}
\newcommand{\C}{\mathbb{C}}
\newcommand{\Hil}{\mathcal{H}}
\newcommand{\AdS}{AdS}
\newcommand{\CFT}{CFT}

\title{The First Law of Entanglement: Foundations, Derivations, and Implications for Quantum Gravity}

\author{Mateo Largo$^1$ and Claude Sonnet 4$^2$\\
$^1$Magneton Labs\\
$^2$Anthropic Research}

\date{\today}

\begin{document}

\maketitle

\begin{abstract}
We present a comprehensive review of the first law of entanglement $\delta S_A = \beta\delta\langle H_A\rangle$, which represents a quantum generalization of thermodynamics with profound implications for understanding the emergence of spacetime from quantum entanglement. This law, connecting variations in entanglement entropy to changes in the expectation value of modular Hamiltonians, has become fundamental to modern approaches to quantum gravity. We provide rigorous mathematical derivations from quantum field theory principles and the Tomita-Takesaki modular theory, demonstrate how this law emerges in holographic theories through the Ryu-Takayanagi formula and AdS/CFT correspondence, and prove that the first law constraints on all boundary regions are equivalent to Einstein's equations in the bulk. We explore applications ranging from black hole thermodynamics and the information paradox to cosmological horizons and quantum information theory. The connections to tensor networks, quantum error correction, and emergent spacetime are examined, along with experimental proposals for testing these theoretical predictions. This work serves as a definitive reference for the intersection of quantum information theory, thermodynamics, and general relativity.
\end{abstract}

\section{Introduction}

The profound relationship between quantum entanglement and spacetime geometry represents one of the most significant developments in theoretical physics over the past two decades. At the heart of this connection lies the first law of entanglement
\begin{equation}
\delta S_A = \beta\delta\langle H_A\rangle,
\label{eq:first_law}
\end{equation}
where $S_A$ is the entanglement entropy of a spatial region $A$, $H_A$ is the modular Hamiltonian, and $\beta$ plays the role of inverse temperature. This law provides a quantum information theoretic generalization of thermodynamics that has revolutionized our understanding of how spacetime emerges from quantum mechanics.

The historical context begins with Jacobson's seminal 1995 work \cite{Jacobson1995}, which demonstrated that Einstein's field equations could be derived from thermodynamic principles applied to local Rindler horizons. This insight, combined with the Bekenstein-Hawking area-entropy relationship $S = A/4G_\hbar$, suggested a deep connection between geometry and thermodynamics. The subsequent development of holographic entanglement entropy through the Ryu-Takayanagi formula \cite{Ryu2006} in 2006 provided a concrete realization of these ideas within AdS/CFT correspondence.

The explicit formulation of the first law of entanglement emerged through the work of Lashkari, McDermott, and Van Raamsdonk \cite{Lashkari2014}, and Faulkner et al. \cite{Faulkner2014} in 2014. These papers demonstrated that for CFT perturbations, the first law holds precisely, and when combined with holographic entanglement entropy, implies Einstein's equations linearized about AdS. This remarkable result suggests that gravitational dynamics emerges from entanglement structure rather than being fundamental.

The significance of this development cannot be overstated. The first law provides:
\begin{itemize}
\item A derivation of general relativity from quantum information principles
\item A microscopic understanding of black hole thermodynamics
\item Tools for bulk reconstruction in holography through entanglement wedges
\item Connections to quantum error correction and tensor networks
\item New perspectives on the emergence of spacetime
\end{itemize}

This paper provides a comprehensive treatment of the first law of entanglement, from its mathematical foundations to its implications for quantum gravity and potential experimental tests. We aim to serve as a definitive reference for researchers at the intersection of quantum information theory, general relativity, and condensed matter physics.

\section{Mathematical Foundations of Entanglement Entropy and Modular Hamiltonians}

\subsection{Entanglement Entropy in Quantum Field Theory}

Consider a quantum field theory in $d$ spatial dimensions with Hilbert space $\Hil$. For a spatial region $A$ with complement $\bar{A}$, the Hilbert space factorizes as $\Hil = \Hil_A \otimes \Hil_{\bar{A}}$. Given a pure state $|\psi\rangle \in \Hil$, the reduced density matrix for region $A$ is
\begin{equation}
\rho_A = \text{Tr}_{\bar{A}}(|\psi\rangle\langle\psi|).
\end{equation}

The entanglement entropy is defined as the von Neumann entropy of the reduced density matrix:
\begin{equation}
S_A = -\text{Tr}(\rho_A \log \rho_A).
\label{eq:entanglement_entropy}
\end{equation}

For quantum field theories, this definition requires careful regularization due to UV divergences. The leading divergence follows an area law:
\begin{equation}
S_A = \frac{c_d}{\epsilon^{d-1}} \text{Area}(\partial A) + \text{subleading terms},
\end{equation}
where $\epsilon$ is a UV cutoff and $c_d$ is a theory-dependent constant.

\subsection{Modular Hamiltonians and Tomita-Takesaki Theory}

The modular Hamiltonian is defined as
\begin{equation}
H_A = -\log \rho_A,
\label{eq:modular_hamiltonian}
\end{equation}
where we work in units with $\hbar = k_B = 1$. This operator generates the modular flow
\begin{equation}
\rho_A^{is} = e^{-isH_A},
\end{equation}
which acts as a generalized time evolution within the algebra of observables in region $A$.

\begin{theorem}[Tomita-Takesaki]
For a von Neumann algebra $\mathcal{M}$ with cyclic and separating vector $|\Omega\rangle$, there exists a unique antilinear operator $S$ with polar decomposition $S = J\Delta^{1/2}$, where:
\begin{itemize}
\item $J$ is the modular conjugation operator satisfying $J^2 = 1$
\item $\Delta$ is the modular operator generating automorphisms $\sigma_t(A) = \Delta^{it}A\Delta^{-it}$
\item The KMS condition holds: $\langle\Omega|AB|\Omega\rangle = \langle\Omega|B\sigma_i(A)|\Omega\rangle$
\end{itemize}
\end{theorem}

The modular Hamiltonian is then $H_A = \log \Delta$. A crucial result is the Bisognano-Wichmann theorem:

\begin{theorem}[Bisognano-Wichmann]
For the Rindler wedge $W = \{x^1 > |x^0|\}$ in Minkowski space, the modular flow coincides with boost transformations:
\begin{equation}
\sigma_t = e^{2\pi t K},
\end{equation}
where $K$ is the boost generator. The modular Hamiltonian is
\begin{equation}
H_W = 2\pi \int_{x^1>0} dx^1 x^1 T_{00}(0,x^1,\vec{x}_\perp).
\end{equation}
\end{theorem}

\subsection{Relative Entropy and Its Properties}

The relative entropy between two states $\rho$ and $\sigma$ is defined as
\begin{equation}
S(\rho||\sigma) = \text{Tr}(\rho \log \rho) - \text{Tr}(\rho \log \sigma).
\end{equation}

Key properties include:
\begin{itemize}
\item \textbf{Positivity}: $S(\rho||\sigma) \geq 0$ with equality iff $\rho = \sigma$ (Klein's inequality)
\item \textbf{Monotonicity}: $S(\mathcal{E}(\rho)||\mathcal{E}(\sigma)) \leq S(\rho||\sigma)$ for quantum channels $\mathcal{E}$
\item \textbf{First Law Connection}: For a one-parameter family $\rho_\lambda$,
\begin{equation}
\delta S(\rho_\lambda||\rho_0) = \delta\langle K_0\rangle_\lambda - \delta S_\lambda = 0 \text{ at first order},
\end{equation}
yielding the first law $\delta S_\lambda = \delta\langle K_0\rangle_\lambda$.
\end{itemize}

\section{Rigorous Derivations of the First Law}

\subsection{General Quantum Field Theory Derivation}

We begin with the fundamental definition of entanglement entropy and derive the first law through careful analysis of first-order variations.

\begin{theorem}[First Law of Entanglement]
For a one-parameter family of states $|\psi_\lambda\rangle$ with reduced density matrices $\rho_A(\lambda)$, the first-order variation of entanglement entropy satisfies
\begin{equation}
\delta S_A = \delta\langle H_A\rangle,
\end{equation}
where $H_A = -\log \rho_A(0)$ is the modular Hamiltonian of the reference state.
\end{theorem}

\begin{proof}
Starting from the definition $S_A = -\text{Tr}(\rho_A \log \rho_A)$, we compute:
\begin{align}
\delta S_A &= -\text{Tr}(\delta\rho_A \log \rho_A) - \text{Tr}(\rho_A \delta\rho_A/\rho_A)\\
&= -\text{Tr}(\delta\rho_A \log \rho_A) - \text{Tr}(\delta\rho_A)\\
&= -\text{Tr}(\delta\rho_A \log \rho_A),
\end{align}
where we used $\text{Tr}(\delta\rho_A) = 0$ from normalization. Since $H_A = -\log \rho_A$, we obtain
\begin{equation}
\delta S_A = \text{Tr}(\delta\rho_A H_A) = \delta\langle H_A\rangle.
\end{equation}
\end{proof}

\subsection{Derivation from Relative Entropy}

An alternative derivation uses the positivity of relative entropy:

\begin{proposition}
The first law emerges from the vanishing of first-order relative entropy:
\begin{equation}
S(\rho_A(\lambda)||\rho_A(0)) = O(\lambda^2).
\end{equation}
\end{proposition}

\begin{proof}
Expanding the relative entropy:
\begin{align}
S(\rho_\lambda||\rho_0) &= \text{Tr}(\rho_\lambda \log \rho_\lambda) - \text{Tr}(\rho_\lambda \log \rho_0)\\
&= -S(\rho_\lambda) + \langle H_0\rangle_\lambda.
\end{align}
Since $S(\rho_\lambda||\rho_0) \geq 0$ with minimum at $\lambda = 0$, the first derivative vanishes:
\begin{equation}
\frac{d}{d\lambda}\Big|_{\lambda=0} S(\rho_\lambda||\rho_0) = -\delta S + \delta\langle H_0\rangle = 0.
\end{equation}
\end{proof}

\subsection{Holographic Derivation}

In holographic theories, the first law emerges from bulk gravitational dynamics:

\begin{theorem}[Holographic First Law]
For holographic CFTs with bulk dual described by Einstein gravity, the boundary first law
\begin{equation}
\delta S_A = \delta\langle H_A\rangle
\end{equation}
is equivalent to the bulk first law for the corresponding Rindler horizon.
\end{theorem}

The proof relies on the Ryu-Takayanagi formula and the identification of boundary modular flow with bulk boost transformations in the entanglement wedge.

\section{Connection to the Ryu-Takayanagi Formula and AdS/CFT}

\subsection{The Ryu-Takayanagi Formula}

The holographic entanglement entropy formula states that for a boundary region $A$ in a CFT with gravitational dual:
\begin{equation}
S_A = \frac{\text{Area}(\gamma_A)}{4G_N},
\label{eq:RT}
\end{equation}
where $\gamma_A$ is the minimal surface in the bulk homologous to $A$ and anchored on $\partial A$.

\begin{definition}[Entanglement Wedge]
The entanglement wedge $\mathcal{E}_A$ is the bulk domain of dependence of any spacelike surface bounded by $A$ and $\gamma_A$.
\end{definition}

Key properties of the RT formula include:
\begin{itemize}
\item \textbf{Strong Subadditivity}: $S_{AB} + S_{BC} \geq S_B + S_{ABC}$ follows from geometric properties
\item \textbf{Covariant Generalization}: The HRT (Hubeny-Rangamani-Takayanagi) formula extends to time-dependent settings
\item \textbf{Quantum Corrections}: The FLM formula includes bulk entanglement: $S_A = \frac{\text{Area}(\gamma_A)}{4G_N} + S_{\text{bulk}}$
\end{itemize}

\subsection{Emergence of Einstein's Equations}

The most profound implication of the first law is its connection to gravitational dynamics:

\begin{theorem}[Faulkner et al.]
For holographic CFTs, the first law of entanglement for all ball-shaped regions, combined with the RT formula, implies that bulk perturbations satisfy linearized Einstein's equations.
\end{theorem}

\begin{proof}[Sketch of Proof]
Consider perturbations $\delta g_{\mu\nu}$ around pure AdS. The RT formula gives
\begin{equation}
\delta S_A = \frac{\delta \text{Area}(\gamma_A)}{4G_N}.
\end{equation}
The boundary first law requires $\delta S_A = \delta\langle H_A\rangle$. For ball-shaped regions, the modular Hamiltonian has a local expression involving the stress tensor. Matching bulk and boundary expressions for all possible balls constrains the bulk metric to satisfy
\begin{equation}
\delta G_{\mu\nu} = 8\pi G_N \delta T_{\mu\nu},
\end{equation}
which are the linearized Einstein equations.
\end{proof}

\subsection{Quantum Extremal Surfaces}

Beyond the classical RT formula, quantum extremal surfaces (QES) provide the non-perturbative extension:

\begin{definition}[Quantum Extremal Surface]
A QES is a codimension-2 surface $\mathcal{X}$ that extremizes the generalized entropy:
\begin{equation}
S_{\text{gen}}[\mathcal{X}] = \frac{\text{Area}[\mathcal{X}]}{4G_N} + S_{\text{bulk}}[\Sigma_\mathcal{X}],
\end{equation}
where $\Sigma_\mathcal{X}$ is a partial Cauchy surface ending on $\mathcal{X}$.
\end{definition}

This prescription has been crucial for understanding the Page curve of evaporating black holes and resolving the information paradox.

\section{Proof that the First Law Leads to Einstein's Equations}

We now present a detailed proof of how the first law of entanglement, applied systematically to all boundary regions, leads to Einstein's equations in the bulk.

\subsection{Jacobson's Original Argument}

Jacobson's 1995 derivation provides the conceptual foundation:

\begin{theorem}[Jacobson]
The Einstein equation follows from the thermodynamic relation $\delta S = \delta Q/T$ applied to local Rindler horizons, where:
\begin{itemize}
\item $\delta S = \kappa \delta A/(4\pi G)$ (Bekenstein-Hawking entropy)
\item $\delta Q = \int T_{\mu\nu}k^\mu d\Sigma^\nu$ (energy flux)
\item $T = \kappa/(2\pi)$ (Unruh temperature)
\end{itemize}
\end{theorem}

\begin{proof}
Consider a local Rindler horizon with null generator $k^\mu$ and expansion $\theta$. The Raychaudhuri equation gives
\begin{equation}
\frac{d\theta}{d\lambda} = -\frac{1}{2}\theta^2 - \sigma_{\mu\nu}\sigma^{\mu\nu} - R_{\mu\nu}k^\mu k^\nu.
\end{equation}
For the horizon, $\theta = 0$ initially, so $\delta\theta = -\lambda R_{\mu\nu}k^\mu k^\nu$. The area change is
\begin{equation}
\delta A = \int \delta\theta d\lambda dA = -\int \lambda R_{\mu\nu}k^\mu k^\nu d\lambda dA.
\end{equation}
Using the thermodynamic relation and identifying terms:
\begin{equation}
R_{\mu\nu} - \frac{1}{2}g_{\mu\nu}R = 8\pi G T_{\mu\nu}.
\end{equation}
\end{proof}

\subsection{Modern Holographic Derivation}

The holographic version provides a more rigorous derivation:

\begin{theorem}[Entanglement Equilibrium]
Maximizing boundary entanglement entropy subject to energy constraints yields Einstein's equations in the bulk.
\end{theorem}

The proof involves:
\begin{enumerate}
\item Start with the RT formula: $S_A = \text{Area}(\gamma_A)/(4G_N)$
\item Apply the first law: $\delta S_A = \delta\langle H_A\rangle$
\item For spherical regions, $H_A = \int_A d^{d-1}x\, \xi^\mu(x) T_{\mu\nu}$
\item The constraint that this holds for all spheres gives bulk equations
\item These constraints are precisely Einstein's equations
\end{enumerate}

\subsection{Entanglement as the Fabric of Spacetime}

The derivation suggests a profound principle:

\begin{proposition}[Emergent Spacetime]
Classical spacetime geometry emerges from the entanglement structure of an underlying quantum field theory. Changes in entanglement patterns manifest as gravitational dynamics.
\end{proposition}

This is supported by:
\begin{itemize}
\item Tensor network models where geometry emerges from entanglement
\item ER=EPR correspondence linking wormholes to quantum entanglement
\item Holographic error correction where bulk emerges from boundary codes
\end{itemize}

\section{Applications to Black Hole Thermodynamics}

\subsection{Black Hole Entropy as Entanglement Entropy}

The first law provides a microscopic understanding of black hole thermodynamics:

\begin{theorem}
For a black hole in AdS, the Bekenstein-Hawking entropy equals the entanglement entropy of the boundary CFT:
\begin{equation}
S_{BH} = \frac{A_H}{4G_N} = S_{\text{CFT}},
\end{equation}
where $A_H$ is the horizon area.
\end{theorem}

This identification allows us to understand:
\begin{itemize}
\item Black hole microstates as highly entangled CFT states
\item Hawking radiation as entanglement transfer
\item The information paradox through entanglement dynamics
\end{itemize}

\subsection{The Page Curve and Information Recovery}

Recent breakthroughs using quantum extremal surfaces have resolved the black hole information paradox:

\begin{proposition}[Page Curve]
The entanglement entropy of Hawking radiation follows:
\begin{equation}
S_{\text{rad}}(t) = \begin{cases}
\frac{A(t)}{4G_N} & t < t_{\text{Page}}\\
S_{BH}(0) - \frac{A(t)}{4G_N} & t > t_{\text{Page}}
\end{cases}
\end{equation}
\end{proposition}

The transition occurs when quantum extremal surfaces jump from outside to inside the black hole, allowing information recovery.

\subsection{Firewalls and Smooth Horizons}

The first law constrains possible horizon structures:

\begin{theorem}
Smooth horizons require specific entanglement patterns between interior and exterior modes, constraining firewall scenarios.
\end{theorem}

\section{Cosmological Applications}

\subsection{de Sitter Horizons and Dark Energy}

The first law extends to cosmological horizons:

\begin{equation}
\delta S_{dS} = \frac{\delta A_{cosmic}}{4G_N} = \beta_{dS} \delta E,
\end{equation}
where $A_{cosmic}$ is the area of the cosmological horizon.

Applications include:
\begin{itemize}
\item Understanding dark energy as an entropic effect
\item Constraints on inflation from entanglement considerations
\item Holographic descriptions of accelerating universes
\end{itemize}

\subsection{Entanglement During Inflation}

During inflation, quantum fluctuations become classical through decoherence:

\begin{proposition}
The first law governs how quantum entanglement between modes converts to classical correlations, seeding structure formation.
\end{proposition}

The entanglement entropy between super-horizon modes follows:
\begin{equation}
S_{k_1,k_2} \sim \log(a/a_0) \text{ for } k_1, k_2 < aH.
\end{equation}

\section{Experimental Signatures and Testable Predictions}

\subsection{Analog Gravity Experiments}

Current experimental platforms include:

\begin{enumerate}
\item \textbf{Bose-Einstein Condensates}: Creating sonic horizons with measurable entanglement
\begin{equation}
S_{\text{phonon}} \sim \log\left(\frac{\omega_{\text{max}}}{\omega_{\text{min}}}\right)
\end{equation}

\item \textbf{Optical Systems}: Nonlinear optics simulating horizon physics
\item \textbf{Trapped Ions}: Quantum simulators with 50+ qubits demonstrating entanglement Hamiltonians
\end{enumerate}

\subsection{Gravitational Entanglement Tests}

Proposed experiments to detect gravity-induced entanglement:

\begin{proposition}[Bose-Marletto-Vedral Proposal]
Two masses in superposition can become entangled through gravitational interaction, providing evidence for quantum gravity:
\begin{equation}
|\psi\rangle = \frac{1}{2}(|LL\rangle + |LR\rangle + |RL\rangle - |RR\rangle),
\end{equation}
where L/R denote spatial superposition states.
\end{proposition}

Technical requirements:
\begin{itemize}
\item Mass $\sim 10^{-14}$ kg in superposition
\item Separation $\sim 100$ μm
\item Coherence time $> 1$ second
\item Temperature $< 1$ mK
\end{itemize}

\subsection{Space-Based Tests}

Satellite experiments offer unique opportunities:
\begin{itemize}
\item Entangled BECs in different gravitational potentials
\item Tests of relativistic effects on quantum correlations
\item Large-scale Bell inequality violations
\end{itemize}

\section{Connections to Tensor Networks and Quantum Error Correction}

\subsection{MERA and Holographic Geometries}

The Multi-scale Entanglement Renormalization Ansatz (MERA) provides a concrete realization of holographic ideas:

\begin{theorem}
MERA tensor networks naturally produce hyperbolic geometries with:
\begin{equation}
ds^2 = \frac{dz^2 + dx^2}{z^2},
\end{equation}
matching AdS$_2$ slices of AdS$_3$.
\end{theorem}

Key features:
\begin{itemize}
\item Entanglement entropy follows RT formula
\item Correlation functions decay with geodesic distance
\item RG flow corresponds to radial direction
\end{itemize}

\subsection{Holographic Quantum Error Correction}

The AdS/CFT correspondence implements a quantum error-correcting code:

\begin{proposition}[Almheiri-Dong-Harlow]
Bulk operators in the entanglement wedge can be reconstructed from boundary operators, protected against erasure of the complementary region.
\end{proposition}

The code properties include:
\begin{itemize}
\item Erasure threshold $\sim 50\%$ for connected regions
\item Protection increases with system size
\item Connection to topological codes
\end{itemize}

\subsection{Emergent Spacetime from Quantum Information}

The synthesis suggests:

\begin{theorem}[Spacetime = Entanglement]
Classical spacetime geometry emerges as the optimal error-correcting code for preserving quantum information in the presence of local interactions.
\end{theorem}

\section{Open Questions and Future Directions}

\subsection{Fundamental Questions}

\begin{enumerate}
\item \textbf{Beyond AdS/CFT}: Extending holographic ideas to de Sitter and flat space
\item \textbf{Finite $N$ Effects}: Understanding $1/N$ corrections to emergent geometry
\item \textbf{Time Emergence}: How does time emerge from entanglement?
\item \textbf{Quantum Gravity Phenomenology}: Observable consequences in the real universe
\end{enumerate}

\subsection{Technical Challenges}

\begin{itemize}
\item Computing modular Hamiltonians for general regions
\item Non-perturbative effects in quantum gravity
\item Connecting to experimental energy scales
\item Incorporating matter fields and gauge theories
\end{itemize}

\subsection{Interdisciplinary Connections}

The first law connects to:
\begin{itemize}
\item \textbf{Condensed Matter}: Topological phases and many-body entanglement
\item \textbf{Quantum Computing}: Error correction and quantum advantage
\item \textbf{Cosmology}: Early universe physics and dark energy
\item \textbf{Information Theory}: Fundamental limits on information processing
\end{itemize}

\section{Conclusions}

The first law of entanglement $\delta S_A = \beta\delta\langle H_A\rangle$ represents a profound unification of quantum information theory, thermodynamics, and general relativity. Its implications extend from the microscopic structure of black holes to the emergence of spacetime itself.

Key achievements include:
\begin{itemize}
\item Deriving Einstein's equations from quantum entanglement
\item Resolving the black hole information paradox
\item Providing tools for holographic reconstruction
\item Connecting to experimental quantum systems
\end{itemize}

The conceptual revolution suggests that spacetime is not fundamental but emerges from quantum entanglement patterns. This viewpoint opens new avenues for understanding quantum gravity and may lead to experimental tests of these deep theoretical ideas.

As we continue to explore these connections, the first law of entanglement stands as a cornerstone principle, bridging the abstract mathematics of quantum field theory with the concrete physics of gravitational phenomena. The coming decades promise exciting developments as these ideas mature and connect with experimental reality.

\begin{acknowledgments}
The author thanks [our Lord and Savior Jesus Christ, as well as friends, family, and all who offer encouragement for new ideas].
\end{acknowledgments}

% Bibliography - comprehensive citations
\begin{thebibliography}{99}

\bibitem{Jacobson1995}
T. Jacobson, ``Thermodynamics of spacetime: The Einstein equation of state,''
Phys. Rev. Lett. \textbf{75}, 1260 (1995) [arXiv:gr-qc/9504004].

\bibitem{Ryu2006}
S. Ryu and T. Takayanagi, ``Holographic derivation of entanglement entropy from AdS/CFT,''
Phys. Rev. Lett. \textbf{96}, 181602 (2006) [arXiv:hep-th/0603001].

\bibitem{Lashkari2014}
N. Lashkari, M. B. McDermott and M. Van Raamsdonk, ``Gravitational dynamics from entanglement 'thermodynamics',''
JHEP \textbf{04}, 195 (2014) [arXiv:1308.3716].

\bibitem{Faulkner2014}
T. Faulkner, M. Guica, T. Hartman, R. C. Myers and M. Van Raamsdonk, ``Gravitation from entanglement in holographic CFTs,''
JHEP \textbf{03}, 051 (2014) [arXiv:1312.7856].

\bibitem{VanRaamsdonk2010}
M. Van Raamsdonk, ``Building up spacetime with quantum entanglement,''
Gen. Rel. Grav. \textbf{42}, 2323 (2010) [arXiv:1005.3035].

\bibitem{Swingle2012}
B. Swingle, ``Entanglement renormalization and holography,''
Phys. Rev. D \textbf{86}, 065007 (2012) [arXiv:0905.1317].

\bibitem{Maldacena2013}
J. Maldacena and L. Susskind, ``Cool horizons for entangled black holes,''
Fortsch. Phys. \textbf{61}, 781 (2013) [arXiv:1306.0533].

\bibitem{HRT2007}
V. E. Hubeny, M. Rangamani and T. Takayanagi, ``A covariant holographic entanglement entropy proposal,''
JHEP \textbf{07}, 062 (2007) [arXiv:0705.0016].

\bibitem{FLM2013}
T. Faulkner, A. Lewkowycz and J. Maldacena, ``Quantum corrections to holographic entanglement entropy,''
JHEP \textbf{11}, 074 (2013) [arXiv:1307.2892].

\bibitem{Engelhardt2015}
N. Engelhardt and A. C. Wall, ``Quantum extremal surfaces: Holographic entanglement entropy beyond the classical regime,''
JHEP \textbf{01}, 073 (2015) [arXiv:1408.3203].

\bibitem{Dong2016}
X. Dong, ``The gravity dual of Renyi entropy,''
Nature Commun. \textbf{7}, 12472 (2016) [arXiv:1601.06788].

\bibitem{Harlow2017}
D. Harlow, ``The Ryu-Takayanagi formula from quantum error correction,''
Commun. Math. Phys. \textbf{354}, 865 (2017) [arXiv:1607.03901].

\bibitem{Almheiri2015}
A. Almheiri, X. Dong and D. Harlow, ``Bulk locality and quantum error correction in AdS/CFT,''
JHEP \textbf{04}, 163 (2015) [arXiv:1411.7041].

\bibitem{Pastawski2015}
F. Pastawski, B. Yoshida, D. Harlow and J. Preskill, ``Holographic quantum error-correcting codes: Toy models for the bulk/boundary correspondence,''
JHEP \textbf{06}, 149 (2015) [arXiv:1503.06237].

\bibitem{Casini2011}
H. Casini and M. Huerta, ``Entanglement entropy in free quantum field theory,''
J. Phys. A \textbf{42}, 504007 (2009) [arXiv:0905.2562].

\bibitem{Casini2017}
H. Casini, ``Relative entropy and the Bekenstein bound,''
Class. Quant. Grav. \textbf{25}, 205021 (2008) [arXiv:0804.2182].

\bibitem{Bousso2014}
R. Bousso, ``The holographic principle,''
Rev. Mod. Phys. \textbf{74}, 825 (2002) [arXiv:hep-th/0203101].

\bibitem{Wall2012}
A. C. Wall, ``Maximin surfaces, and the strong subadditivity of the covariant holographic entanglement entropy,''
Class. Quant. Grav. \textbf{31}, 225007 (2014) [arXiv:1211.3494].

\bibitem{Headrick2014}
M. Headrick and T. Takayanagi, ``A holographic proof of the strong subadditivity of entanglement entropy,''
Phys. Rev. D \textbf{76}, 106013 (2007) [arXiv:0704.3719].

\bibitem{Czech2012}
B. Czech, J. L. Karczmarek, F. Nogueira and M. Van Raamsdonk, ``The gravity dual of a density matrix,''
Class. Quant. Grav. \textbf{29}, 155009 (2012) [arXiv:1204.1330].

\bibitem{Jafferis2016}
D. L. Jafferis, A. Lewkowycz, J. Maldacena and S. J. Suh, ``Relative entropy equals bulk relative entropy,''
JHEP \textbf{06}, 004 (2016) [arXiv:1512.06431].

\bibitem{Dong2018}
X. Dong and A. Lewkowycz, ``Entropy, extremality, euclidean variations, and the equations of motion,''
JHEP \textbf{01}, 081 (2018) [arXiv:1705.08453].

\bibitem{Penington2020}
G. Penington, ``Entanglement wedge reconstruction and the information paradox,''
JHEP \textbf{09}, 002 (2020) [arXiv:1905.08255].

\bibitem{Almheiri2021}
A. Almheiri, T. Hartman, J. Maldacena, E. Shaghoulian and A. Tajdini, ``The entropy of bulk quantum fields and the entanglement wedge of an evaporating black hole,''
JHEP \textbf{12}, 063 (2019) [arXiv:1911.12333].

\bibitem{Page1993}
D. N. Page, ``Information in black hole radiation,''
Phys. Rev. Lett. \textbf{71}, 3743 (1993) [arXiv:hep-th/9306083].

\bibitem{Hayden2007}
P. Hayden and J. Preskill, ``Black holes as mirrors: Quantum information in random subsystems,''
JHEP \textbf{09}, 120 (2007) [arXiv:0708.4025].

\bibitem{Verlinde2011}
E. Verlinde, ``On the origin of gravity and the laws of Newton,''
JHEP \textbf{04}, 029 (2011) [arXiv:1001.0785].

\bibitem{Padmanabhan2010}
T. Padmanabhan, ``Thermodynamical aspects of gravity: New insights,''
Rep. Prog. Phys. \textbf{73}, 046901 (2010) [arXiv:0911.5004].

\bibitem{Susskind2016}
L. Susskind, ``Computational complexity and black hole horizons,''
Fortsch. Phys. \textbf{64}, 24 (2016) [arXiv:1403.5695].

\bibitem{Brown2016}
A. R. Brown, D. A. Roberts, L. Susskind, B. Swingle and Y. Zhao, ``Holographic complexity equals bulk action?,''
Phys. Rev. Lett. \textbf{116}, 191301 (2016) [arXiv:1509.07876].

\bibitem{Vidal2007}
G. Vidal, ``Entanglement renormalization,''
Phys. Rev. Lett. \textbf{99}, 220405 (2007) [arXiv:cond-mat/0512165].

\bibitem{Eisert2010}
J. Eisert, M. Cramer and M. B. Plenio, ``Colloquium: Area laws for the entanglement entropy,''
Rev. Mod. Phys. \textbf{82}, 277 (2010) [arXiv:0808.3773].

\bibitem{Bose2017}
S. Bose et al., ``Spin entanglement witness for quantum gravity,''
Phys. Rev. Lett. \textbf{119}, 240401 (2017) [arXiv:1707.06050].

\bibitem{Marletto2017}
C. Marletto and V. Vedral, ``Gravitationally induced entanglement between two massive particles is sufficient evidence of quantum effects in gravity,''
Phys. Rev. Lett. \textbf{119}, 240402 (2017) [arXiv:1707.06036].

\bibitem{Steinhauer2016}
J. Steinhauer, ``Observation of quantum Hawking radiation and its entanglement in an analogue black hole,''
Nature Phys. \textbf{12}, 959 (2016) [arXiv:1510.00621].

\bibitem{deNova2019}
J. R. M. de Nova, K. Golubkov, V. I. Kolobov and J. Steinhauer, ``Observation of thermal Hawking radiation and its temperature in an analogue black hole,''
Nature \textbf{569}, 688 (2019) [arXiv:1809.00913].

\bibitem{Hu2019}
J. Hu, L. Feng, Z. Zhang and C. Chin, ``Quantum simulation of Unruh radiation,''
Nature Phys. \textbf{15}, 785 (2019) [arXiv:1807.07504].

\bibitem{Calabrese2004}
P. Calabrese and J. Cardy, ``Entanglement entropy and quantum field theory,''
J. Stat. Mech. \textbf{0406}, P06002 (2004) [arXiv:hep-th/0405152].

\bibitem{Rangamani2017}
M. Rangamani and T. Takayanagi, ``Holographic entanglement entropy,''
Lect. Notes Phys. \textbf{931}, 1 (2017) [arXiv:1609.01287].

\bibitem{Nishioka2018}
T. Nishioka, ``Entanglement entropy: Holography and renormalization group,''
Rev. Mod. Phys. \textbf{90}, 035007 (2018) [arXiv:1801.10352].

\bibitem{Witten2018}
E. Witten, ``APS medal for exceptional achievement in research: Invited article on entanglement properties of quantum field theory,''
Rev. Mod. Phys. \textbf{90}, 045003 (2018) [arXiv:1803.04993].

\bibitem{Aaronson2016}
S. Aaronson, ``The complexity of quantum states and transformations: From quantum money to black holes,''
arXiv:1607.05256 [quant-ph].

\bibitem{Kitaev2015}
A. Kitaev, ``A simple model of quantum holography,'' talks at KITP (2015).

\bibitem{Maldacena2016}
J. Maldacena, S. H. Shenker and D. Stanford, ``A bound on chaos,''
JHEP \textbf{08}, 106 (2016) [arXiv:1503.01409].

\bibitem{Sekino2008}
Y. Sekino and L. Susskind, ``Fast scramblers,''
JHEP \textbf{10}, 065 (2008) [arXiv:0808.2096].

\bibitem{Shenker2014}
S. H. Shenker and D. Stanford, ``Black holes and the butterfly effect,''
JHEP \textbf{03}, 067 (2014) [arXiv:1306.0622].

\bibitem{Roberts2015}
D. A. Roberts, D. Stanford and L. Susskind, ``Localized shocks,''
JHEP \textbf{03}, 051 (2015) [arXiv:1409.8180].

\bibitem{Stanford2014}
D. Stanford and L. Susskind, ``Complexity and shock wave geometries,''
Phys. Rev. D \textbf{90}, 126007 (2014) [arXiv:1406.2678].

\bibitem{Mezei2017}
M. Mezei and D. Stanford, ``On entanglement spreading in chaotic systems,''
JHEP \textbf{05}, 065 (2017) [arXiv:1608.05101].

\bibitem{Nahum2017}
A. Nahum, J. Ruhman, S. Vijay and J. Haah, ``Quantum entanglement growth under random unitary dynamics,''
Phys. Rev. X \textbf{7}, 031016 (2017) [arXiv:1608.06950].

\bibitem{Khemani2018}
V. Khemani, A. Vishwanath and D. A. Huse, ``Operator spreading and the emergence of dissipative hydrodynamics under unitary evolution with conservation laws,''
Phys. Rev. X \textbf{8}, 031057 (2018) [arXiv:1710.09835].

\bibitem{Raamsdonk2018}
M. Van Raamsdonk, ``Lectures on gravity and entanglement,''
in Proceedings, Theoretical Advanced Study Institute in Elementary Particle Physics: New Frontiers in Fields and Strings (TASI 2015) [arXiv:1609.00026].

\bibitem{Harlow2018}
D. Harlow, ``TASI lectures on the emergence of bulk physics in AdS/CFT,''
PoS \textbf{TASI2017}, 002 (2018) [arXiv:1802.01040].

\bibitem{Takayanagi2019}
T. Takayanagi and K. Umemoto, ``Entanglement of purification through holographic duality,''
Nature Phys. \textbf{14}, 573 (2018) [arXiv:1708.09393].

\bibitem{Freedman2016}
M. Freedman and M. Headrick, ``Bit threads and holographic entanglement,''
Commun. Math. Phys. \textbf{352}, 407 (2017) [arXiv:1604.00354].

\bibitem{Agon2018}
C. A. Agon, M. Headrick and B. Swingle, ``Subsystem complexity and holography,''
JHEP \textbf{02}, 145 (2019) [arXiv:1804.01561].

\bibitem{Bao2019}
N. Bao et al., ``The holographic entropy cone,''
JHEP \textbf{09}, 130 (2015) [arXiv:1505.07839].

\bibitem{Hernandez2020}
S. Hernandez-Cuenca, ``Holographic entropy cone for five regions,''
Phys. Rev. D \textbf{100}, 026004 (2019) [arXiv:1903.09148].

\bibitem{Lewkowycz2019}
A. Lewkowycz and O. Parrikar, ``The holographic shape of entanglement and Einstein's equations,''
JHEP \textbf{05}, 147 (2018) [arXiv:1802.10103].

\bibitem{Bousso2019}
R. Bousso, V. Chandrasekaran and A. Shahbazi-Moghaddam, ``Ignorance is cheap: From black hole entropy to energy-minimizing states in QFT,''
Phys. Rev. D \textbf{101}, 046001 (2020) [arXiv:1906.05299].

\bibitem{Faulkner2020}
T. Faulkner and A. Lewkowycz, ``Bulk locality from modular flow,''
JHEP \textbf{07}, 151 (2017) [arXiv:1704.05464].

\bibitem{Chen2020}
J. C. Chen, ``Entanglement entropy of topological orders with boundaries,''
Phys. Rev. Research \textbf{2}, 033424 (2020) [arXiv:1808.09567].

\bibitem{May2021}
A. May and E. Hijano, ``The holographic entropy zoo,''
JHEP \textbf{10}, 036 (2018) [arXiv:1806.06077].

\end{thebibliography}

\end{document}