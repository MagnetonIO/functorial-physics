%%%%%%%%%%%%%%%%%%%%%%%%%%%%%%%%%%%%%%%%%%%%%%%%%%%%%%%%%%%%%%%%%%%%%%%%
%%                                                                    %%
%%  Unifying Physics via Category and Topos Theory:                   %%
%%  The Functorial Schrödinger Equation                               %%
%%                                                                    %%
%%  arXiv-style LaTeX version                                         %%
%%                                                                    %%
%%%%%%%%%%%%%%%%%%%%%%%%%%%%%%%%%%%%%%%%%%%%%%%%%%%%%%%%%%%%%%%%%%%%%%%%

\documentclass[12pt]{article}

\usepackage{amsmath,amssymb,amsfonts,amsthm}
\usepackage{hyperref}
\usepackage{geometry}
\usepackage{graphicx}
\usepackage{cite}
\usepackage{slashed}
\usepackage{bm}
\usepackage{stmaryrd}   % For extra symbols if needed

\geometry{
  a4paper,
  total={6in, 9in},
  left=1.0in,
  right=1.0in,
  top=1.2in,
  bottom=1.2in,
}

\hypersetup{
    colorlinks=true,
    linkcolor=blue,
    citecolor=blue,
    urlcolor=blue
}

\newtheorem{theorem}{Theorem}[section]
\newtheorem{definition}[theorem]{Definition}
\newtheorem{lemma}[theorem]{Lemma}
\newtheorem{proposition}[theorem]{Proposition}
\newtheorem{corollary}[theorem]{Corollary}

%%%%%%%%%%%%%%%%%%%%%%%%%%%%%%%%%%%%%%%%%%%%%%%%%%%%%%%%%%%%%%%%%%%%%%%%
%% Title, Author, and Abstract
%%%%%%%%%%%%%%%%%%%%%%%%%%%%%%%%%%%%%%%%%%%%%%%%%%%%%%%%%%%%%%%%%%%%%%%%

\title{\textbf{Unifying Physics via Category and Topos Theory: \\ 
The Functorial Schrödinger Equation}}

\author{
  \textbf{Matthew Long} \\
  \emph{Magneton Labs}
}

\date{\today}

\begin{document}

\maketitle

\begin{abstract}
This paper surveys and refines the foundations of quantum theory by leveraging tools from category theory and topos theory. We present how quantum states and observables can be recast in a presheaf topos, highlight how time evolution and the Schrödinger equation admit a powerful functorial formulation, and explore advanced mathematical structures such as derived functors, homotopical and motivic methods, and higher symmetries. We then connect these ideas to quantum gravity (loop quantum gravity, spin foam models), quantum information (categorical quantum circuits, ZX-calculus), and topological quantum field theories (TQFTs), ultimately motivating a deeper and more structural understanding of quantum mechanics and its potential extensions. Finally, we propose a \emph{Functorial Schrödinger Equation} that unifies dynamical evolution, measurement, and quantum logic within one categorical framework.
\end{abstract}

\tableofcontents

%%%%%%%%%%%%%%%%%%%%%%%%%%%%%%%%%%%%%%%%%%%%%%%%%%%%%%%%%%%%%%%%%%%%%%%%
%% 1. A Topos-Theoretic Framework for Quantum Mechanics
%%%%%%%%%%%%%%%%%%%%%%%%%%%%%%%%%%%%%%%%%%%%%%%%%%%%%%%%%%%%%%%%%%%%%%%%

\section{A Topos-Theoretic Framework for Quantum Mechanics}

\textbf{Categorical Structures and Quantum States.} Quantum physics can be recast in terms of category theory by identifying appropriate categorical structures that mirror quantum logic and state spaces. A key idea is to consider a category of contexts---for example, the poset of all commutative subalgebras of a quantum system’s operator algebra---and then consider the presheaf topos of sets on this context category. In this topos (a category of functors from the context category to $\mathbf{Sets}$), one finds a canonical object called the \emph{spectral presheaf}, denoted $\Sigma$. For each context (commuting set of observables) $V$, $\Sigma(V)$ is defined as the spectrum of that commutative subalgebra (all its pure states). Intuitively, $\Sigma$ assigns to each classical perspective $V$ the set of possible outcomes as if the system were constrained to that context. This spectral presheaf plays the role of a quantum state space in the topos---the analogue of a classical phase space or state set. 

Crucially, however, it behaves differently from a classical state space: in particular, it has no global elements (no single section choosing one outcome in all contexts), a fact equivalent to the Kochen--Specker theorem. Physically, this means one cannot assign definite values to all quantum observables at once; there is no ``hidden'' point in $\Sigma$ corresponding to a simultaneous microstate for all observables. This captures the essence of quantum contextuality in categorical form.

\medskip
\textbf{Topos Theory and Quantum Logic.} By working in the topos of presheaves, we effectively replace classical Boolean logic with an internal logic more suited to quantum theory. The spectral presheaf $\Sigma$, despite lacking points, is a rich object whose lattice of subobjects encodes quantum propositions. In the topos approach pioneered by Isham and Döring, one associates to each physical proposition (like ``observable $A$ has value in $\Delta$'') a corresponding subobject of $\Sigma$. Truth values of propositions are not simple binary numbers but elements of an internal Heyting algebra (an intuitionistic logic) within the topos.

A key construction is \emph{daseinisation}, which maps a projection operator $P$ (a yes--no quantum proposition) to a clopen subobject $d(P)$ of $\Sigma$. In essence, $d(P)$ is the best approximation of proposition $P$ in each classical context; it ``throws'' the quantum proposition into the classical worlds of each context. This mapping $P \mapsto d(P)$ lets one define a logical structure where each quantum proposition has an image as a (context-dependent) classical proposition. The collection of all such contextualised propositions (the clopen subobjects of $\Sigma$) is a Heyting algebra, obeying distributivity but not the law of the excluded middle. In other words, quantum logic in the topos is intuitionistic: a proposition may neither be true nor false absolutely (reflecting the absence of global truth values), aligning with the idea that quantum truth is contextual. This offers a logically consistent ``neo-realist'' picture of quantum theory where each observable always has a value, but the value exists as a varying entity across different classical contexts rather than a single global number.

\medskip
\textbf{Sheaf-Theoretic States and Observables.} Within this topos framework, both quantum states and observables acquire natural sheaf-theoretic representations. Since $\Sigma$ has no global sections (no microstates), one cannot represent a pure quantum state as a single point. Instead, Isham and Döring introduce ``truth objects'' or state objects in the topos. A truth object is essentially a subobject of $\Sigma$ (or a collection of local sections) that consistently assigns truth values to propositions, analogous to a state assigning truth to ``$A\in \Delta$.''

Meanwhile, an observable (a self-adjoint operator $A$) can be identified with a natural transformation from the spectral presheaf $\Sigma$ to a quantity-value object. For example, one can show that each self-adjoint operator $A$ defines a continuous real-valued function on $\Sigma$---essentially a morphism from $\Sigma$ to the presheaf of real intervals. Concretely, in the Heunen--Landsman--Spitters formulation (often called \emph{Bohrification}), the topos-theoretic analogue of a quantum phase space is a locale (a point-free topological space) internal to the topos, and every quantum observable $A$ corresponds to a locale map from that internal phase space to a space of real numbers. In this setting, states on the original $C^*$-algebra (normal states or density matrices) correspond to probability measures (valuations) on the internal spectrum $\Sigma$. The traditional Born rule emerges from an internal pairing between states and propositions: given a state (valuation) $\mu$ and a proposition subobject $E$, one can define an internal truth value $\mu(E)$ in the topos. This pairing is mathematically an evaluation map that assigns to each state $\mu$ a truth-value in $[0,1]$ for each proposition, reproducing the probabilities or expectation values as in standard quantum mechanics.

Notably, because logic is intuitionistic, $\mu(E)=1$ or $0$ represent certainty about $E$, while intermediate values might be handled by a separate layer of probability or by considering $\mu$ as a fuzzy truth assignment. In summary, quantum theory becomes akin to a classical theory internal to a topos: the spectral presheaf serves as a state space object, propositions form a logical algebra of subobjects, states are internal measures/valuations, and observables act as (internal) real-valued functions on the state space. This entire construction is functorial and grounded in category theory, providing new insights into quantum foundations---for instance, it bypasses no-go results by working with a generalized notion of ``point'' and truth. The approach has been successfully related to traditional quantum structures (recovering the usual predictions of quantum mechanics) while offering a new axiomatic foundation that could be extendable to theories like quantum gravity where a global, classical perspective is absent.

%%%%%%%%%%%%%%%%%%%%%%%%%%%%%%%%%%%%%%%%%%%%%%%%%%%%%%%%%%%%%%%%%%%%%%%%
%% 2. Functorial Reformulation of the Schrödinger Equation
%%%%%%%%%%%%%%%%%%%%%%%%%%%%%%%%%%%%%%%%%%%%%%%%%%%%%%%%%%%%%%%%%%%%%%%%

\section{Functorial Reformulation of the Schrödinger Equation}

\textbf{State Evolution as a Functor.}
The time evolution in quantum mechanics, governed by the Schrödinger equation, can be reformulated in category-theoretic terms by treating time as an (ordered) category and the evolution as a functor. In the Schrödinger picture, a quantum state $\lvert\psi(t)\rangle$ evolves under a one-parameter family of unitary operators $U(t) = e^{-iHt}$ generated by the Hamiltonian $H$. We can view this as a representation of the time-translation group (or the monoid of non-negative times) on the state space. Categorically, a group can be seen as a one-object category (the single object ``the system,'' with morphisms corresponding to elements of the time group), and a representation is exactly a functor from that one-object category to $\mathbf{Vect}$ (or $\mathbf{Hilb}$) that sends the single object to the Hilbert space $\mathcal{H}$ of the system and each morphism $t$ to the linear operator $U(t)$ on $\mathcal{H}$.

In this sense, ordinary quantum mechanics is essentially a 1-dimensional functorial field theory: it can be understood as a (1+0)-dimensional quantum field theory, meaning a monoidal functor from the category of 1D cobordisms (intervals representing time evolutions) to the category of vector spaces (or Hilbert spaces). Concretely, consider the category $1\mathrm{Cob}$ whose objects are points (interpreted as ``instantaneous time slices'' or boundary states) and whose morphisms are 1-dimensional intervals (time evolutions) between these points. A quantum system defines a functor
\[
   Z: 1\mathrm{Cob} \; \to \; \mathbf{Hilb}
\]
by assigning to the single point object the Hilbert space $\mathcal{H}$ of states, and to each time-interval (morphism) the corresponding unitary evolution operator on $\mathcal{H}$. Composition of time intervals (gluing two intervals end to end) corresponds to composition of the unitary operators, reflecting the group law $U(t_2 + t_1) = U(t_2)U(t_1)$. This functorial viewpoint is powerful: it encodes the Schrödinger equation’s solution $\lvert\psi(t)\rangle = U(t)\lvert\psi(0)\rangle$ as functorial action on states, and it generalizes readily. If time has additional structure (say, a partial order or branching), one could consider a category of time slices and their relationships, and a functor into a state category that respects those relations.

\smallskip
Composition of two cobordisms (spacetime ``pair-of-pants'' surfaces $M$ and $N$) into a single cobordism $M \cup N$, and the corresponding composition of linear maps $U_M: H_{\partial M_{\rm in}} \to H_{\partial M_{\rm out}}$ and $U_N: H_{\partial N_{\rm in}} \to H_{\partial N_{\rm out}}$ into the composed map $U_{M\cup N} = U_N \circ U_M$. In a functorial quantum theory, gluing spacetime pieces corresponds to composing the associated operators. This illustrates how categorical composition encodes physical time evolution.

The above idea is in line with the functorial quantum field theory (FQFT) program. In general, an $n$-dimensional quantum field theory can be axiomatized as a functor from an $n$-Cobordism category (with $(n-1)$-dimensional manifolds as objects and $n$-dimensional cobordisms as morphisms) to a target category of algebraic objects (vector spaces, Hilbert spaces, or more elaborate structures). Standard quantum mechanics is the case $n=1$: each 0-dimensional boundary (point) is assigned a state space, and each 1-dimensional time evolution interval is assigned a linear map (the propagator or unitary) between state spaces, such that composition of intervals maps to composition of maps. This functorial reformulation is not just a mathematical curiosity---it provides a globally consistent view of quantum evolution. It ensures automatically that if you chop an evolution from time 0 to $T$ into two segments $[0,t_1]$ and $[t_1,T]$, the overall evolution is the composite of the two segment evolutions. This is built into the functoriality $Z([0,T]) = Z([t_1,T] \circ [0,t_1]) = Z([t_1,T]) \circ Z([0,t_1])$, which is just the statement of the semigroup property of solutions of the Schrödinger equation. In more abstract terms, the Schrödinger equation itself (as a first-order differential equation in time) can be seen as specifying an infinitesimal generator for this functor, ensuring $\tfrac{d}{dt}U(t) = -\tfrac{i}{\hbar} H U(t)$.

\medskip
\textbf{Categorical Wavefunctions and Natural Transformations.} Describing wavefunction evolution categorically also sheds light on the wavefunction itself. In the conventional picture, a wavefunction at time $t$ is a vector $\lvert \psi(t)\rangle \in \mathcal{H}$. In category terms, since a state is a morphism from the monoidal unit object $\mathbb{C}$ to the state space object (in $\mathbf{Hilb}$, a pure state $\lvert\psi\rangle$ can be identified with the morphism $\lambda \mapsto \lvert\psi\rangle \cdot \lambda$), the time-evolving wavefunction can be seen as a morphism that changes with time.

One way to formalize this is to consider a functor 
\[
   \mathcal{S}: \mathcal{T} \to \mathbf{Hilb},
\]
where $\mathcal{T}$ is a small category representing instants or time steps (possibly the natural numbers for discrete time or an index category for continuous time). $\mathcal{S}$ assigns to each time instant $t$ the Hilbert space $\mathcal{H}$ (possibly the same for all $t$ if the state space is time-independent) and to each time step $(t \to t')$ the propagator $U(t'-t)$ as before. Then a particular solution $\lvert \psi(t)\rangle$ of the Schrödinger equation can be regarded as a \emph{natural transformation} between this functor $\mathcal{S}$ and a trivial functor $I$ that picks out a reference state (say the initial state at $t=0$) for all times. Intuitively, the natural transformation provides, for each time $t$, a map 
\[
   \eta_t: I(t) \; \longrightarrow \; \mathcal{S}(t)
\]
which is exactly the state $\lvert\psi(t)\rangle$ in $\mathcal{H}$, and the naturality condition ensures consistency with time evolution: for each $t \to t'$, 
\[
   \mathcal{S}(t \to t') \,\circ\, \eta_t 
   \;=\; \eta_{t'} \,\circ\, I(t \to t'),
\]
which boils down to $U(t'-t) \lvert\psi(t)\rangle = \lvert\psi(t')\rangle$.

This is a bit abstract, but it shows that the wavefunction’s evolution can itself be treated as a morphism in a higher-level category, and the Schrödinger equation imposes that this morphism be natural (consistent with the functor structure of time). In more concrete \emph{categorical quantum mechanics} frameworks (such as Abramsky--Coecke’s dagger-compact categories approach), one often identifies a state $\lvert\psi\rangle$ with a morphism $I \xrightarrow{\,|\psi\rangle\,} A$ where $A$ is the object representing the quantum system. Evolution then acts as composition on these state-morphisms. This view is neatly diagrammatic: in string diagrams for monoidal categories, a state is drawn as a ``cap'' (an input from nothing to the system wire), and its evolution by an operator $U$ is drawn by connecting that cap through a box $U$ to yield a new state.

\medskip
\textbf{Enriched Functorial Structures (Beyond Kolmogorov Probability).} The traditional description of quantum evolution involves probabilities (via the Born rule for measurement outcomes). Category theory allows generalizing these probabilistic aspects in enriched or synthetic ways. One promising direction is the use of \emph{Markov categories} and their quantum analogues. Markov categories are categories where morphisms can be thought of as stochastic maps, abstracting the notion of probability distributions and stochastic processes in categorical terms. Recently, researchers have defined \emph{quantum Markov categories}, which extend the Markov category framework to encompass quantum probability (noncommutative probabilities, or density matrices and completely positive maps). In such a setting, one can view a quantum state not just as a vector, but as a morphism that outputs a probability amplitude or outcome distribution. Quantum processes (including unitary evolution, measurement, decoherence, etc.) become morphisms in a category that is enriched over a suitable base capturing probabilistic mixing or quantum superpositions.

For example, instead of enriching in $\mathbf{Set}$ (as ordinary categories do), one might enrich the category of quantum systems in $\mathbf{Vect}$ or $\mathbf{Hilb}$ (to keep track of complex amplitudes) or in a category of effect algebras (to track probabilities in a generalized way). This leads to structures like dagger symmetric monoidal categories of quantum channels or $C^*$-enriched categories. The functors in such enriched categories can encode not only deterministic evolution but also stochastic or quantum uncertainties.

A concrete instance is the idea of viewing the entire Schrödinger evolution as a functor between probabilistic theories. The Schrödinger equation itself conserves the $L^2$ norm (total probability 1), and can be seen as a special case of a stochastic evolution that happens to be unitary (and hence invertible). By embedding standard quantum mechanics into a broader category of probabilistic theories, one can compare it to other theories (classical probability, or hypothetical post-quantum theories) in one unified formalism. Work in categorical probability shows that one can handle Bayesian updating, inference, and even entropy in terms of universal properties in Markov categories. Translating Schrödinger dynamics into this language means treating the unitary $U(t)$ as a morphism that not only sends pure states to pure states, but also carries along the statistical structure (for example, sending each density operator $\rho$ to $U(t)\,\rho\,U(t)^\dagger$). Indeed, one can define a functor between categories of state-spaces: from the category of quantum states at time $0$ (with channels as morphisms) to the category of quantum states at time $t$, induced by conjugation with $U(t)$. This functorial picture naturally extends to mixed states and completely positive maps. In enriched terms, it is a 2-functor (since we are dealing with categories of processes) that preserves the probabilistic structure.

Such generalized, enriched functorial formulations are not just of abstract interest---they are relevant in contexts like quantum information theory, where one wants to compare classical and quantum processes on the same footing, or in the study of operational theories (general probabilistic theories) where quantum theory is one point in a landscape of possible theories. By recasting the Schrödinger equation in these functorial terms, we gain the ability to replace the notion of probability measure by more general weightings or logical values, which is essential in some approaches to quantum foundations (like topos theory above, where truth values replace probabilities in a fundamental way). This functorial perspective is also the stepping stone to incorporating more sophisticated mathematics into quantum dynamics, such as homotopy (for gauge symmetries) or higher categorical symmetries, as we discuss next.

%%%%%%%%%%%%%%%%%%%%%%%%%%%%%%%%%%%%%%%%%%%%%%%%%%%%%%%%%%%%%%%%%%%%%%%%
%% 3. Derived Hamiltonians and Advanced Mathematical Tools
%%%%%%%%%%%%%%%%%%%%%%%%%%%%%%%%%%%%%%%%%%%%%%%%%%%%%%%%%%%%%%%%%%%%%%%%

\section{Derived Hamiltonians and Advanced Mathematical Tools}

\textbf{Homotopical Quantum Mechanics.} Traditional quantum mechanics mostly uses the language of linear algebra and functional analysis, but emerging approaches apply homotopy theory and higher category theory to capture situations with complex topological or gauge-theoretic structures. In \emph{homotopical quantum mechanics}, one considers that the spaces of states or observables might not just be plain vector spaces or algebras, but carry additional layers (chain complexes, homotopies, etc.) that encode equivalences and symmetries beyond the usual sense. For example, in gauge theories or quantum gravity, many physical degrees of freedom are redundant (connected by local symmetries). These can be handled by BRST/BV complex techniques, which introduce ghost fields and view the physical observables as the cohomology of a certain differential. Category-theoretically, this suggests treating the quantum system’s algebra of observables not as a single algebra object, but as part of a complex or derived object in a homotopy category.

Recent work on homotopical or model-categorical quantum field theory formalizes this: one assigns to each region of spacetime not just an algebra $A$ of observables, but a chain complex of observables (or an $E_\infty$-algebra) that encodes the entire space of fields and their gauge redundancies. The result is that many of the constructions become functors into the category $\mathbf{Ch}$ of chain complexes or \emph{Spectra} (in stable homotopy theory) rather than into $\mathbf{Vect}$. Higher category theory also enters through $n$-categories of cobordisms: as Baez and Dolan and others have outlined, an \emph{extended} quantum field theory assigns data not just to full spacetimes and their boundaries, but also to corners, points, etc., requiring an $n$-functor from an $n$-category of cobordisms to an $n$-category of algebraic objects. Applying this to quantum mechanics (which is 1-dimensional QFT) suggests considering 2-categorical structures for processes---for instance, treating unitary evolutions as 1-morphisms and perhaps introducing 2-morphisms as equivalences between evolutions (this could relate to path homotopies or quantum channels connecting unitaries). In homotopical terms, one might say two Hamiltonians $H_0$ and $H_1$ (with associated propagators) are homotopic if there is a continuous interpolation between them---this could be captured by a 2-morphism in an $\infty$-category of quantum systems, potentially encoding an equivalence of dynamics.

While such ideas are still being developed, they promise a formulation of quantum mechanics that naturally incorporates gauge equivalences (treating physically equivalent Hamiltonians as connected) and topological features of parameter space (like Berry phases, which are fundamentally homotopy-theoretic: they depend on loops in parameter space).

\medskip
\textbf{Derived Functors in Hamiltonian Dynamics.} When we speak of refining the Hamiltonian formulation with derived functors and spectral structures, we are invoking tools from homological algebra and algebraic geometry. A derived functor typically arises in contexts where one needs to adjust a construction (like a functor between categories) to account for objects that aren’t strictly nice (e.g.\ not exact). In physics, a parallel idea is the need to deal with ``rough'' spaces of states or observables---e.g.\ the space of solutions to equations of motion might be a complicated geometric space (possibly a stack) that is best understood with derived geometry. 

For instance, the path integral in quantum field theory can be thought of as a kind of pushforward (integration) of a measure on a space of field histories to a constant (to get a number or an amplitude). In difficult cases (especially with gauge symmetries or infinite-dimensional spaces), this pushforward needs to be handled in a derived sense---taking into account singularities or lack of proper fiber bundles. Recent work by mathematicians like Kevin Costello and Jacob Lurie on derived algebraic geometry and quantum field theory treats the critical points of action functionals as derived intersections, leading to constructions like the derived critical locus which underpins the BV formalism.

In simpler quantum-mechanical terms, one could consider a scenario where the Hamiltonian $H$ is part of a family or is time-dependent: then solving the Schrödinger equation is akin to a homological calculation (summing an infinite Dyson series, etc.), which might be expressed as a derived functor on a chain complex of time-ordered interactions. Although that is a heuristic analogy, more concretely, one can attach to a Hamiltonian system a spectral complex---for example, consider the spectral decomposition of $H$: it gives a resolution of the identity $I = \sum_n \lvert \phi_n \rangle \langle \phi_n \rvert$. This is reminiscent of a direct sum decomposition, which one can regard as a functor from the category of representations of the observable algebra to $\mathbf{Vect}$. If the spectrum is continuous or has multiplicities, one often uses tools like rigged Hilbert spaces, which again have a flavor of derived objects (distributions as continuous linear functionals, etc.). 

By incorporating spectral structures, we mean allowing these decompositions and functional calculi to be part of the categorical formalism. For example, one could envision an \emph{eigensheaf} that assigns to each energy value the corresponding eigenspace. The Hamiltonian itself then acts as a kind of functor on this sheaf (by shifting phases $e^{-iEt}$ on each eigenspace). In topos theory, something analogous happens: the Gel’fand transform yields an isomorphic image of a commutative algebra as functions on the spectrum. For noncommutative $H$, we don’t have a straightforward space, but we can use internal spectra in a topos or consider a noncommutative geometry perspective where one works with categories of modules or complexes over the noncommutative algebra. This is closely related to derived categories in algebraic geometry (as appear, for instance, in the study of D-branes in string theory, where a brane is akin to an object in a derived category of coherent sheaves). Translating such ideas to quantum mechanics, one could attempt to model a quantum system’s state space as a derived motive---an object that carries not just vector space structure but also additional cohomological data that might correspond to something like a space of trajectories or histories.

\medskip
\textbf{Motivic and Derived Structures Connecting to Quantum Theory.} One of the more speculative but intriguing directions is the connection of quantum mechanics to \emph{motives} and advanced algebraic geometry. A motive is an abstraction from algebraic geometry that captures the common essence of various cohomology theories; remarkably, motivic structures have been found to enter quantum physics in subtle ways. For example, in deformation quantization (the formal quantization of Poisson algebras), Kontsevich and others observed that the freedom in quantization (the different possible star-products) is related to the action of the Grothendieck--Teichmüller group, which is essentially a motivic Galois group. This indicates that at a perturbative level, choices in quantization correspond to elements of a fundamental group of the moduli of surfaces, connecting to number theory and motives. Additionally, Connes and Marcolli have found that the process of renormalization in quantum field theory has an interpretation involving a ``cosmic Galois group,'' again related to motivic structures.

On the other hand, geometric quantization---especially in modern formulations via higher category theory---can be seen as a two-step derived functor: a pull-push (adjoint) pair on a correspondence of spaces. In a cohesive topos approach (as described by Schreiber), quantization is realized by taking an action functional $S: \mathcal{X} \to \mathbb{A}^1$ (from a space of fields $\mathcal{X}$ to the line, say) and forming a correspondence $\ast \leftarrow \mathcal{X} \xrightarrow{S}\mathbb{A}^1$, then applying a ``pull'' along $\mathcal{X} \xrightarrow{S}\mathbb{A}^1$ (which gives something like the exponential $e^{iS}$ viewed as a cocycle in twisted cohomology) and then a ``pushforward'' along $\mathcal{X}\to \ast$ (which integrates over the fields). The result is a kind of motivic path integral---essentially an index in a certain cohomology theory. This procedure reveals pure motives (in the sense of algebraic geometry) in what we interpret as quantum amplitudes.

While this is more directly tied to field theory, it casts light on even quantum-mechanical path integrals (which are 1-dimensional). It suggests that the time-evolution operator can be seen as a kind of ``pushforward'' of a phase $e^{-iHt/\hbar}$, and if we treat that pushforward in a derived fashion (accounting for all critical points and fluctuations properly), we might interpret the result in terms of special functions or numbers with motivic significance. In simpler terms, even solving the Schrödinger equation $i\hbar\, \frac{d}{dt}\lvert \psi\rangle = H\,\lvert \psi\rangle$ can be formally written as $\lvert \psi(t)\rangle = e^{-iHt/\hbar}\lvert \psi(0)\rangle$. If one expands $e^{-iHt}$ as a power series and looks at the integrals involved (for example, the Feynman--Dyson series in an interaction picture), one encounters integrals that in some cases evaluate to values of zeta functions or multiple zeta values (which are associated with motives). Thus, there is an implicit motivic footprint even in non-relativistic quantum mechanics when treated perturbatively.

The emerging synthesis is that using derived algebraic geometry (with its notions of higher stacks, $\infty$-categories, and derived functors like $\mathbf{R}\!\mathrm{Hom}$ or $\mathbf{L}\!\mathrm{Tensor}$) one could potentially recast quantum mechanics in a form that naturally interfaces with quantum geometry and gravity. For instance, in approaches to quantum gravity like the topos one or spin-foam models, spacetime itself is not fixed, and one deals with a spectrum of possible geometries. A derived-categorical formulation of quantum mechanics might be able to handle such a superposition or integration over geometries as part of the formalism. Though these ideas are cutting-edge and not yet a completed theory, the integration of motives, homotopy, and categories into quantum theory is expected to yield deeper symmetry insights (like dualities and hidden symmetries in the space of solutions) and possibly new computational techniques. 

In summary, introducing derived and homotopical tools into quantum mechanics can be seen as promoting the static Hilbert space and operators into dynamic objects in an $\infty$-category, where solutions to Schrödinger’s equation, equivalences of quantum evolutions, and even quantization itself become functorial or adjoint processes. This rich structure could illuminate connections between quantum mechanics and other areas (topology, number theory) and pave the way for a consistent theory of quantum gravity where spacetime and quantum states are described in one unified categorical language.

%%%%%%%%%%%%%%%%%%%%%%%%%%%%%%%%%%%%%%%%%%%%%%%%%%%%%%%%%%%%%%%%%%%%%%%%
%% 4. Applications to Quantum Gravity, Quantum Information, and TQFTs
%%%%%%%%%%%%%%%%%%%%%%%%%%%%%%%%%%%%%%%%%%%%%%%%%%%%%%%%%%%%%%%%%%%%%%%%

\section{Applications to Quantum Gravity, Quantum Information, and TQFTs}

\textbf{Quantum Gravity in Functorial/Topos Language.} One of the original motivations for the topos approach was indeed quantum gravity. In quantum gravity and cosmology, we often deal with situations where there is no fixed classical background (no external time or observer). The topos framework, with its internal logic and state space, offers a kind of background-independent formulation of quantum theory---the theory is ``internal'' to a category and does not presume an external classical world to define truth or state. Isham and Döring envisioned that a suitable choice of topos could handle quantum spacetime theory. For example, in loop quantum gravity (LQG), the configuration space of the theory is very non-classical (based on discrete geometry states). Efforts have been made to apply topos theory to LQG, by constructing a topos of presheaves where contexts might be choices of a graph or a finite set of loops on which the quantum states (cylindrical functions) are defined. Indeed, a Topos model for LQG was developed in which the state object $\Sigma$ is a functor on a category of commutative subalgebras of the holonomy-flux algebra (the algebra of basic LQG observables). Each such context corresponds roughly to ``looking at the quantum geometry with a finite resolution'' (akin to a lattice or a graph); the spectral presheaf then assigns to each such context its classical spectrum (which could be interpreted as a set of eigenvalues of geometry operators in that context).

Remarkably, the \emph{Bohrification} procedure (named after Niels Bohr’s idea of using classical perspectives) allows one to convert the full noncommutative algebra of LQG into a commutative locale internal to a topos. The outcome is an internal quantum geometry: points in this locale correspond to ``neo-realist'' states of spacetime geometry that are not single classical geometries but consistent collections of classical data across contexts. This approach has shown, for instance, how to recover diffeomorphism invariance and other symmetry properties in the topos setting. In more category-theoretic terms, one might say that quantum gravity states (like spin network states in LQG) form a category (or 2-category) rather than a set, and functorial methods are natural to organize them. John Baez’s work on spin foams and higher gauge theory exemplifies this: spin network states (graphs labeled by group representations) can be seen as morphisms in a category (with edges as wires carrying representations and nodes as intertwiners), and the spin foam (a quantum history connecting spin networks) is then a 2-morphism in a 2-category of cobordisms with labeled boundaries. This is essentially a TQFT-like structure for quantum gravity, where the functor assigns to each boundary (space) a state space and to each foam (spacetime) an amplitude. Category theory thus provides the natural language for gluing spacetime regions and for stating the constraint of general covariance (independence of how you chop spacetime). For example, the general boundary formulation of quantum theory (developed by Oeckl and others) is explicitly category-theoretic: it treats any boundary (of any shape, not just at equal times) as a system and assigns amplitudes to spacetime regions with those boundaries, forming a functor that is local and compositional.

In quantum cosmology, topos theory has been used to address the ``problem of time'' by replacing the external Boolean logic (which fails in timeless quantum universes) with an internal logic where one can recover a notion of truth that is not pinned to an external clock. In sum, whether via topos or via higher category, quantum gravity benefits from a functorial viewpoint in which states and processes are organized in a network respecting gluing of spacetime. The new formalism can reformulate key aspects like constraints and invariants in a categorical way. For instance, a Hamiltonian or diffeomorphism constraint in gravity (which annihilates physical states) can be seen as an endomorphism in a category whose kernel defines the physical sub-object of states; verifying the constraint algebra might translate to checking that certain diagrams commute (the constraints as natural transformations that must satisfy relations). This language also interfaces well with quantum topology---e.g.\ using TQFT techniques to model simplified quantum gravity in lower dimensions, or using knot categories to analyze entanglement of space (since spin networks are essentially graphs, their entanglement might be studied with category-theoretic invariants of graphs). The upshot is that category theory and topos theory provide powerful unifying frameworks that can accommodate the lack of fixed spacetime and the contextual nature of geometry in quantum gravity, while remaining consistent with ordinary quantum mechanics in appropriate limits.

\medskip
\textbf{Quantum Information and Functorial Quantum Circuits.} Category theory has already had a profound impact on quantum information science, through the development of \emph{categorical quantum mechanics} (CQM). In CQM, one models finite-dimensional quantum processes using dagger-compact categories, which abstract the algebraic properties of Hilbert spaces and linear maps in terms of objects and morphisms that can be drawn as string diagrams. This has led to high-level diagrammatic calculi (like the ZX-calculus) for quantum circuits and protocols. The functorial and topos-based refinements of quantum mechanics naturally enhance our understanding of quantum information as well. 

In these approaches, a quantum circuit is not just a sequence of operations but can be understood as a morphism in a monoidal category of processes. The inputs and outputs of the circuit (quantum bits, for instance) are objects, and the gates (unitaries, measurements) are morphisms, with the tensor product capturing parallel composition and composition of morphisms capturing sequential application. One striking result from categorical quantum information is that fundamental protocols like quantum teleportation can be derived as purely diagrammatic consequences of the axioms of a dagger-compact category. This means the protocol does not require heavy calculation once the problem is set up in the right category---it follows from the functorial nature of copying and deleting information in that category.

By reformulating the Schrödinger equation functorially, we integrate the dynamic evolution into this picture: rather than treating time evolution as separate from quantum circuits, we can include continuous time gates or Hamiltonian evolution segments as morphisms as well. For example, a continuous unitary evolution $e^{-iHt}$ can be seen as a limiting case of a sequence of quantum gates; category theory can handle this either by extending to dagger-compact closed categories with exponentials (to represent continuous transformations) or by working in an ind-category that approximates the continuous by discrete. In practical terms, the functorial approach clarifies structures like entanglement: entangled states are morphisms from the monoidal unit to a bipartite system, and their properties (like being non-separable) correspond to the morphism not factorizing through simpler objects.

The new formalism could allow one to generalize quantum information beyond standard probability. For instance, one might consider topos-inspired quantum computing, where instead of qubits being two-dimensional Hilbert spaces over $\mathbb{C}$, they could be objects in a topos (which could incorporate some logical or security constraints), and processing them would be functors between topoi. Though speculative, this could have applications in scenarios like quantum computing on encrypted data or in formal verification of quantum software---essentially bringing logical and categorical semantics into the hardware level.

Another application is categorical quantum error correction: quantum error correcting codes can be seen as certain subobjects in the category of Hilbert spaces with extra structure (like a code is a subspace $C \subset H$ that is fixed by some noise operators). Category theory provides tools like adjoint functors to describe encoding and decoding as adjoint processes, or coalgebras for comonads to represent the repetitive structure of error correction. These abstractions may lead to more systematic constructions of codes or new insights (for example, viewing a code state as a limit of an infinite tensor network, which is a categorical limit). In quantum networking, functoriality can describe how local quantum operations (on nodes of a network) compose to global protocols. We see, then, that integrating functorial physics with quantum information yields a highly modular, compositional view of quantum algorithms and processes. It becomes easy to reason about entangled resources, resource theories, and asymptotic limits using categorical terms like monoidal closure and functorial dilution (e.g.\ asymptotic states can be understood via filtered colimits in a category of states). 

This high-level approach complements the circuit-level description and connects to logic (via categorical logic) and computer science (via monoidal categories and functors being analogous to type constructors and programs). In practice, tools from this approach such as the ZX-calculus have already simplified reasoning about quantum circuits (e.g.\ simplifying certain quantum computations can be done by diagram rewrites rather than matrix algebra). The continued development of the functorial Schrödinger framework could similarly simplify reasoning about quantum algorithms that involve continuous time evolution or algorithms in analog quantum simulators (where $e^{-iHt}$ is the main operation). By encompassing these within one categorical model, we get a unified language to talk about both circuit-based and Hamiltonian-based quantum computation.

\medskip
\textbf{Topological Quantum Field Theories and Extended Symmetries.} The functorial framework for quantum processes is fundamentally the framework of topological quantum field theories (TQFTs) as introduced by Atiyah and Segal. A TQFT is defined as a symmetric monoidal functor 
\[
   Z: \mathrm{Cob}_n \;\to\; \mathbf{Vect}
\]
(or $\mathbf{Hilb}$ for a unitary theory) that assigns to each $(n-1)$-dimensional closed manifold a vector space (the state space) and to each $n$-dimensional cobordism a linear map (the ``evolution'' or partition function). By expressing quantum mechanics in similar language, we make a direct connection: ordinary quantum mechanics can be viewed as a 1-dimensional TQFT (sometimes humorously called ``0+1 dimensional QFT''). This not only provides elegant proofs of quantum properties but also opens up generalizations. For example, one can consider time that itself has topology---like a periodic time (giving quantum mechanics on a circle, related to thermal field theory)---and treat it with the TQFT axioms. The categorical viewpoint naturally incorporates topological phases and order: in condensed matter and quantum computing, one studies systems where the global phase of matter is described by a TQFT (e.g.\ the effective theory of a fractional quantum Hall state is a Chern--Simons TQFT). Functorial physics allows describing quantum processes that are insensitive to certain deformations (hence topological) as special functors that factor through a quotient of the cobordism category.

Moreover, the notion of \emph{extended} TQFT (where one assigns data also to lower-dimensional manifolds like points, lines, etc.) aligns with higher-category approaches to quantum symmetries. ``Extended quantum symmetries'' refers to symmetries that are not just groups acting on states, but higher groups or categories acting on higher structures of the theory. For instance, a 1-form symmetry in field theory is not a symmetry acting on point-like operators but on line operators (e.g.\ the Wilson loops); such a symmetry is better described as a categorical symmetry (a functor between category of line operators) rather than an ordinary group action. Recent developments have formalized these ideas: one finds that certain quantum systems have symmetry operators forming a fusion category instead of a group, which is then termed a \emph{categorical symmetry}. Examples include the symmetry of the toric code model, which involves string operators that obey fusion rules similar to a category of representations of a quantum group.

By integrating these into the functorial framework, we can model an extended symmetry as an extra layer on the functor. For instance, a 2D TQFT with a categorical symmetry might be described by a functor that actually targets not just $\mathbf{Vect}$, but something like the category of module categories over a fusion category (so that symmetry actions correspond to autoequivalences of those module categories). In simpler terms, the functorial Schrödinger picture can be augmented so that it is equivariant under a higher symmetry: instead of 
\[
   Z: \mathrm{Cob}_n \,\to\, \mathbf{Hilb},
\]
we might have $Z$ landing in a category on which a fusion category (symmetry) acts, and $Z$ respects that action. This allows treatment of non-invertible symmetries (like duality defects or anyon types) on the same footing as ordinary symmetries.

The topos approach, too, can accommodate extended symmetries: since a topos has an internal logic, a symmetry could appear as an automorphism of the topos or its objects. For example, a symmetry exchanging two observables would appear as a natural transformation on the spectral presheaf (if it permutes contexts appropriately). All these are facilitated by the categorical reformulation, because categories naturally handle operations on operations (morphisms between morphisms are 2-morphisms, capturing symmetry-of-symmetry). We see this concretely in quantum information as well, where quantum channels form a category, and a supermap (mapping one channel to another, like a symmetry or a higher operation) is then a functor or a 2-morphism in a 2-category of processes. In topological terms, extended TQFT provides a classification of such higher operations via the cobordism hypothesis, which is a deep result linking $n$-functors to fully dualizable objects in symmetric monoidal $n$-categories. In physics language, it tells us how specifying the ``local data'' (like the Hilbert space on a sphere, and operators on a disk, etc.) determines the global theory. Similarly, by specifying the ``local'' pieces of quantum mechanical structure (states, observables, symmetries at infinitesimal level), a functorial framework can in principle build up the entire theory consistently.

This is extremely helpful in quantum gravity (where one expects an extended TQFT structure in a yet-unknown form) and in quantum computation (where fault-tolerant logical operations often have an interpretation as extended symmetries of the code space). Thus, the new formalism, anchored in category and topos theory, has wide-ranging applications: it recasts quantum gravity in a language that handles background-independence and contextuality; it offers quantum information a high-level, diagrammatic toolbox to design and verify protocols; and it connects quantum mechanics to TQFT and topological quantum computing (where qubits are quasi-particles with braided categorical statistics). By pushing these ideas, one may even find experimental signatures of the new viewpoint---for instance, categorical symmetries predict certain degeneracies or selection rules in spectra that ordinary symmetry might miss, and those could be tested in engineered quantum systems or novel materials.

%%%%%%%%%%%%%%%%%%%%%%%%%%%%%%%%%%%%%%%%%%%%%%%%%%%%%%%%%%%%%%%%%%%%%%%%
%% 5. Beyond Probability: Axiomatic and Structural Extensions
%%%%%%%%%%%%%%%%%%%%%%%%%%%%%%%%%%%%%%%%%%%%%%%%%%%%%%%%%%%%%%%%%%%%%%%%

\section{Beyond Probability: Axiomatic and Structural Extensions}

\textbf{Beyond Kolmogorov Probability.} One of the motivations for using category and topos theory in quantum mechanics is to transcend the usual probabilistic interpretation and find more general or structural notions of ``state'' and ``measurement.'' In standard quantum theory, probabilities arise via the Born rule, which is formulated within the Hilbert space framework and ultimately grounded in classical measure theory (the probability of obtaining result $e$ in state $\psi$ is $\lvert P_e \psi \rvert^2$, resembling a measure on the spectrum of an observable). However, quantum phenomena have prompted exploration of generalized probability theories. R.\ Sorkin, for example, proposed a hierarchy of ``extended probability'' sum-rules, showing that classical probability (which is additive on exclusive events) is just the first in a hierarchy, the next level of which yields quantum mechanics (with the two-slit interference pattern requiring a second-order additivity, but not third-order). In Sorkin’s quantum measure theory, one assigns a generalized measure $\mu$ to sets of histories such that $\mu(A\cup B\cup C)$ satisfies a certain grade-2 additivity (vanishing of the triple-interference term). This framework can recover the rule that quantum probabilities are quadratic in amplitudes as a natural consequence, and it predicts that higher-order interferences (like a triple-slit interference term beyond the sum of pairs) should be zero---a prediction which has been experimentally tested (so far confirming the absence of third-order interference).

This is an example of moving ``beyond probability'' in the sense of replacing the Kolmogorov axioms with something broader that still respects known physics. Category theory accommodates such generalizations gracefully: one can consider categories of histories and a measure as a functor from the category’s sigma-algebra (or a suitable structure) to $[0,1]$ that only partially preserves sums. In a topos, indeed, probabilities are replaced by truth values in a richer logic. Instead of assigning each proposition a number in $[0,1]$, one assigns it an element of an internal truth-object (a subobject of the state object $\Sigma$) representing ``to what extent the proposition is true.'' In Isham--Döring’s approach, a state is a truth assignment that picks, for each context (classical perspective) $V$, a point in the spectrum $\Sigma(V)$ such that these points are compatible between contexts. This is essentially a global section of a sub-sheaf of $\Sigma$ (since a genuine global section doesn’t exist). Such a state assignment yields an internal probability: for any observable $A$ and Borel set $\Delta$, one can ask if ``$A\in\Delta$'' is true in that state. The answer will not be simply yes or no in general, but a truth value in the topos’s logic. If the topos is $\mathbf{Sets}$, this reduces to a characteristic function (1 or 0), but in the quantum topos it is an element of a Heyting algebra that encodes partial truth. Thus we have extended the idea of probability (a $[0,1]$ number) to a much richer concept of contextual truth degree. This approach is still made such that if one were to interpret that truth value externally, one would recover usual probabilities in agreement with Born’s rule, but internally it avoids needing an external probability measure at all---everything is deterministic but in a many-valued logic sense. This resonates with ideas in empirical logic and the effect algebra approach to quantum theory: instead of probabilities, one works with \emph{effects} (operators between 0 and 1) which represent fuzzy predicates.

\medskip
\textbf{Axiomatic Extensions of Quantum Mechanics.} Category theory encourages us to formulate axioms in terms of universal properties or algebraic structures rather than specific numbers. For instance, one can axiomatize quantum theory as the theory of a noncommutative probability space (a von Neumann algebra) together with a state (a normal linear functional). Going beyond this could mean axiomatizing a class of theories via category-theoretic conditions. One prominent line is the \emph{generalized probabilistic theories} (GPT) framework, which describes physical theories by the convex sets of states and effects satisfying certain axioms (like no faster-than-light signaling). Category theory provides a language for GPTs too (via ordered vector spaces or compact convexity categories). Another approach is to maintain ``physical relevance'' (i.e.\ recover quantum and classical cases) while extending conceptual foundations.

One such extension is \emph{Effectus theory}, a recently developed categorical logic for reasoning about quantum and classical systems in one framework. In effectus theory, one does not assume probabilities \emph{a priori}; instead, one assumes that each system $X$ comes with a set (actually an ordered monoid) of effects $E(X)$ (abstract yes-no tests), which form an effect algebra. States are defined as certain morphisms (called stateless or effect-normalized morphisms $I \to X$) and there is an intrinsic notion of the validity of an effect in a state given by a scalar in $E(I)$. This reproduces the Born rule internally: an effect $e$ applied to state $\omega$ yields a scalar $\omega \models e$ which in classical cases is a probability and in quantum cases is $\langle \psi \lvert E \lvert \psi \rangle$ for some effect operator $E$. The power of the effectus approach is that it doesn’t assume the state spaces are Hilbert spaces or probability simplices---those appear as special models of the theory. Instead, it formulates axioms like ``each predicate has a complement,'' ``there is a sequential product for effects,'' etc., some of which hold in quantum, some in classical, some in both. By tweaking these axioms, one can explore hypothetical theories (maybe one where double negation = identity, forcing classical logic, or one where some distributivity law fails like in quantum). The category-theoretic mindset encourages finding minimal sets of axioms that ensure desirable physical properties (like no signaling or existence of entangled states) and then exploring the landscape of possible theories satisfying them. This could lead to discovering either new theories or confirming that quantum theory is essentially unique under those conditions. For example, one famous result along these lines is that compact dagger-monoidal categories with certain pure-state decomposability axioms force the category to be equivalent to finite-dimensional Hilbert spaces (thus essentially characterizing quantum theory).

Axiomatic extensions also involve combining different theories: a topos could combine classical and quantum parts (yielding something like a hybrid logic for quantum-classical interaction), or one could consider multi-sorted systems with classical and quantum types in a single categorical universe.

\medskip
\textbf{Physical Relevance and Experimental Prospects.} It’s important that these abstract extensions remain tied to physics. One way this is ensured is by demanding that the usual quantum mechanics be a special case or limit of the new formalism. For instance, the topos approach is constructed so that if one chooses the topos of $\mathbf{Sets}$, one recovers classical physics; if one chooses the topos of presheaves on contexts, one recovers quantum physics. Thus, any test of quantum mechanics (Bell tests, interference experiments, etc.) can be reinterpreted in the new formalism and should give the same predictions. However, the new formalism might suggest new experiments in cases where standard quantum theory is silent or ambiguous. For example, the third-order interference experiment mentioned earlier: classical theory predicts zero third-order term, standard quantum also predicts zero (as a rule rather than a fundamental axiom), but an extended probabilistic theory could allow it. Experiments confirmed the term is zero to within experimental error, thereby supporting the quantum measure theory approach that also enforced it to zero. Another example is that some topos-based researchers talk about measuring truth values of propositions rather than probabilities---in practice, this could relate to weak measurements or contextuality tests.

In quantum information, categorical approaches have led to practical algorithms (e.g.\ circuit simplification with ZX-calculus) which one could view as an experimental/computational validation---if a categorical simplification preserves the semantics and an actual quantum device produces the same output after fewer gates, it validates the approach. In quantum gravity, it’s harder to get direct experiments, but a theoretical validation is \emph{consistency}: for example, showing that a topos-based quantization of cosmology avoids certain singularities or anomalies would be a form of validation that the traditional approach struggled with.

There is also an interesting possibility: by formulating quantum theory in new mathematical terms, we might discover new physical regimes where those terms are natural. For instance, intuitionistic logic might find a place in scenarios with limited knowledge or finite precision---perhaps relevant for coarse-grained quantum measurements or finite quantum computing where you deliberately restrict certain classical reasoning for security. Similarly, higher symmetries predicted by categorical symmetry theory might be observed in exotic materials or metamaterials designed to have fusion-rule-like excitations (as in topological quantum computing anyons). As a final note, these extensions strive to be conservative in that they do not contradict known physics but expand the conceptual landscape. By doing so, they not only aim to solve outstanding foundational issues (like reconciling quantum theory with gravity, or explaining the quantum-to-classical transition in new terms) but also equip physicists with new tools---functors, sheaves, higher algebra---which could solve practical problems (like simplifying computations or classifying new phases of matter). The rigorous mathematical derivations behind these ideas ensure that while the language is abstract, the results are concrete: every new equivalence or duality proven categorically translates to a statement about differential equations or operators that can be checked. Going forward, this line of research holds promise for a more unified theory of quantum processes---one that treats physics not as disparate equations, but as a coherent collection of functors, objects, and morphisms, beautifully mirroring the interconnected structure of reality itself.

%%%%%%%%%%%%%%%%%%%%%%%%%%%%%%%%%%%%%%%%%%%%%%%%%%%%%%%%%%%%%%%%%%%%%%%%
%% 6. The Functorial Schrödinger Equation (Final Section)
%%%%%%%%%%%%%%%%%%%%%%%%%%%%%%%%%%%%%%%%%%%%%%%%%%%%%%%%%%%%%%%%%%%%%%%%

\section{The Functorial Schrödinger Equation}

We conclude by emphasizing the central role of the \emph{Functorial Schrödinger Equation} as a unifying concept.  In the standard Schrödinger picture, one solves
\[
   i\hbar\, \frac{d}{dt} \lvert \psi(t)\rangle \;=\; H\,\lvert \psi(t)\rangle,
\]
with $\lvert \psi(0)\rangle$ given.  A strong case can be made that this is precisely the condition that a certain functor---representing time evolution as a one-parameter group of automorphisms (or a monoidal representation of the time-category)---is ``infinitesimally generated'' by $H$.  

To see this more explicitly, let us represent \emph{time} as a monoidal category $\mathcal{T}$ with one object $\bullet$ and morphisms $t:\bullet\to\bullet$ labeled by nonnegative real numbers $t\ge 0$, where composition is addition of times.  A \emph{functor} $F:\mathcal{T}\to \mathbf{Hilb}$ then assigns the same Hilbert space $\mathcal{H}$ to the single object and assigns to each morphism $t$ the operator $F(t)=U(t)$, a (possibly) unitary evolution on $\mathcal{H}$.  The usual group/semigroup property says $F(t_1+t_2)=F(t_1)\circ F(t_2)$.

Requiring $F(t)$ to satisfy
\[
   \frac{d}{dt}\, F(t) \;=\; -\tfrac{i}{\hbar}\,H\,F(t)
\]
can be recognized as an \emph{infinitesimal statement} about the derivative of the functor with respect to the ``time'' parameter in $\mathcal{T}$.  In more precise terms, one uses a differential extension of $\mathcal{T}$ (turning it into a topological category or a groupoid with structure allowing derivatives) to require that $H$ be the generator of the representation $t\mapsto e^{-iHt/\hbar}$.  

From a natural-transformation viewpoint, the wavefunction $\lvert\psi(t)\rangle$ is a component of a natural transformation $\eta$ between the trivial functor $I$ (that sends every morphism in $\mathcal{T}$ to $\mathrm{id}:\mathbb{C}\to\mathbb{C}$) and the evolution functor $F$.  The Schrödinger equation ensures precisely that the component $\eta_t:\mathbb{C}\to \mathcal{H}$ transforms by $F(t)$ consistently with the initial condition.  

Thus, in the \emph{Functorial Schrödinger Equation}, the usual operator relation 
\[
   U(t)\;=\;\exp\!\Bigl(-\tfrac{i}{\hbar}Ht\Bigr)
\]
becomes the exponentiated representation of the time-category, and $\lvert\psi(t)\rangle=U(t)\lvert\psi(0)\rangle$ is the induced natural transformation that selects the solution wavefunction at each time.  By embedding this into broader categorical constructs (like those described in Sections~1--5), one obtains a platform for analyzing not only single-system time evolution but also the gluing of spacetimes (cobordisms), composite processes (in quantum information), and generalized measures or truth values (in quantum foundations).  

In short, the ``functorialization'' of the Schrödinger equation accomplishes more than an elegant rewriting: it systematizes the fundamental postulates of quantum mechanics---unitarity, composition, measurement---within a single mathematical environment.  This vantage point paves the way for consistent unification with approaches in algebraic geometry, topological field theory, and quantum gravity, strengthening the core principle that quantum dynamics itself is best viewed as \emph{morphisms in a category}, whose shape and structure reveal deeper symmetries and foundational insights.

\bigskip

%%%%%%%%%%%%%%%%%%%%%%%%%%%%%%%%%%%%%%%%%%%%%%%%%%%%%%%%%%%%%%%%%%%%%%%%
%% References
%%%%%%%%%%%%%%%%%%%%%%%%%%%%%%%%%%%%%%%%%%%%%%%%%%%%%%%%%%%%%%%%%%%%%%%%

\begin{thebibliography}{99}

\bibitem{isham_doring_topos}
C.~Isham and A.~Döring,
\newblock \emph{A Topos Foundation for Theoretical Physics},
\newblock J.\ Math.\ Phys.\ \textbf{49}, 053515 (2008).

\bibitem{heunen_landsman_spitters_bohr}
C.~Heunen, N.~Landsman, and B.~Spitters,
\newblock \emph{A Topos for Algebraic Quantum Theory: Bohrification},
\newblock Commun.\ Math.\ Phys.\ \textbf{291}, 63--110 (2009).

\bibitem{abramsky_coecke}
S.~Abramsky and B.~Coecke,
\newblock \emph{A Categorical Semantics of Quantum Protocols},
\newblock in \emph{Proceedings of the 19th Annual IEEE Symposium on Logic in Computer Science}, 415--425 (2004).

\bibitem{baez_dolan}
J.~Baez and J.~Dolan,
\newblock \emph{Higher-Dimensional Algebra and Topological Quantum Field Theory},
\newblock J.\ Math.\ Phys.\ \textbf{36}, 6073--6105 (1995).

\bibitem{schreiber_cohesive}
U.~Schreiber,
\newblock \emph{Differential Cohomology in a Cohesive Topos},
\newblock \url{https://arxiv.org/abs/1310.7930}.

\bibitem{connes_marcolli}
A.~Connes and M.~Marcolli,
\newblock \emph{Noncommutative Geometry, Quantum Fields and Motives},
\newblock American Mathematical Society Colloquium Publications, Vol.~55, 2008.

\bibitem{kontsevich_formal}
M.~Kontsevich,
\newblock \emph{Deformation Quantization of Poisson Manifolds},
\newblock Lett.\ Math.\ Phys.\ \textbf{66}, 157--216 (2003).

\bibitem{sorkin_qmeasure}
R.~Sorkin,
\newblock \emph{Quantum Mechanics as Quantum Measure Theory},
\newblock Mod.\ Phys.\ Lett.\ A \textbf{9}, 3119--3127 (1994).

\bibitem{jacobs_effectus}
B.~Jacobs,
\newblock \emph{New Directions in Categorical Logic, for Classical, Probabilistic and Quantum Logic},
\newblock \url{https://arxiv.org/abs/1305.4582}.

\bibitem{coecke_kissinger}
B.~Coecke and A.~Kissinger,
\newblock \emph{Picturing Quantum Processes: A First Course in Quantum Theory and Diagrammatic Reasoning},
\newblock Cambridge University Press, 2017.

\end{thebibliography}

\end{document}
