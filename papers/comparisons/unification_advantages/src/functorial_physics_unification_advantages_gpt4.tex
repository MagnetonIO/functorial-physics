\documentclass[11pt,a4paper]{article}
\usepackage[utf8]{inputenc}
\usepackage[T1]{fontenc}
\usepackage{geometry}
\geometry{margin=1in}
\usepackage{amsmath,amssymb,amsthm}
\usepackage{graphicx}
\usepackage{hyperref}
\usepackage{booktabs}
\usepackage{authblk}
\usepackage{enumitem}

\title{Advantages of Functorial Physics over Conventional Frameworks}

\author[1]{Matthew Long}
\author[2]{ChatGPT (OpenAI Assistant)}
\affil[1]{Yoneda AI Research}
\affil[2]{OpenAI Research Assistant}

\date{\today}

\begin{document}

\maketitle

\begin{abstract}
Functorial Physics offers a high-level, categorical reformulation of physics where physical processes are represented as functors between structured categories. This framework surpasses traditional models like Quantum Field Theory, String Theory, and Loop Quantum Gravity by achieving greater unification, composability, and interpretability. We summarize the key advantages and present a comparative overview.
\end{abstract}

\section{Introduction}
Traditional physical frameworks rely on rigid mathematical structures (e.g., Hilbert spaces, Riemannian manifolds, or differential equations) and often treat spacetime as a fixed background. Functorial Physics reconceives all physical processes through the lens of category theory—specifically as functors between structured categories—enabling a powerful and composable representation of dynamics, symmetries, and quantum-classical relations.

\section{Summary of Advantages}

\begin{itemize}[leftmargin=2em]
  \item \textbf{Composability}: Functorial processes can be composed like software functions.
  \item \textbf{Unification}: General Relativity, Quantum Mechanics, and Logic unify through categorical structures.
  \item \textbf{Interpretability}: Clear semantics for time, causality, and measurement via limits and adjoints.
  \item \textbf{Formality}: Built on rigorous foundations of higher category theory and topos theory.
  \item \textbf{Extensibility}: Naturally supports extensions to quantum error correction, decoherence, and AI modeling.
\end{itemize}

\section{Comparative Table}

\begin{table}[h!]
\centering
\begin{tabular}{@{}llcccccc@{}}
\toprule
\textbf{Feature} & \textbf{Functorial Physics} & QFT & String Theory & LQG & Axiomatic Physics \\
\midrule
Foundational Language & Higher Category Theory & Functional Analysis & Worldsheet Geometry & Spin Networks & Set Theory / Hilbert Spaces \\
Background Spacetime & Emergent (\textbf{\texttimes}) & Fixed (\checkmark) & Fixed (10D) & Background-Independent (\texttimes) & Typically Fixed (\checkmark) \\
Quantum-Classical & Natural via Adjointness & Ad hoc & Requires Compactification & Still Unclear & Not Fully Captured \\
Unification & \textbf{Yes} (GR, QM, Logic) & \texttimes & Partial & Partial & Fragmented \\
Compositionality & \textbf{Yes} & \texttimes & \texttimes & Limited & Informal \\
Information-Theoretic & Central & Secondary & Emergent & Partial & Not Intrinsic \\
Error Correction & Derived Naturally & External Postulates & Complex Mechanisms & Weakly Defined & Rarely Addressed \\
Geometrical Flexibility & Abstract / Topos-Based & Riemannian & Calabi--Yau & Spin Foams & Varies \\
AI Integration & Structured, Diagrammatic & Difficult & Complex & Sparse & Possible \\
\bottomrule
\end{tabular}
\caption{Comparison of Functorial Physics with major existing physical theories.}
\end{table}

\section{Conclusion}
Functorial Physics reframes fundamental physics as a compositional, information-theoretic structure rooted in category theory. Its advantages span formal unification, rigorous semantics, and better support for automated reasoning. As physics increasingly intersects with computation, AI, and information theory, Functorial Physics provides a powerful and future-proof framework.

\end{document}